\section{Calibration and Event Reconstruction}

\subsection{Gain Calibration of PMTs} 

We use ET 9823QKB PMTs that have a dark current up to 1000~nA as specified by the manufacturer. At a
nominal high voltage of $\sim$2,400~V, it was impossible to observe a single photoelectron peak due to the
high dark noise rate. Therefore we developed a special procedure for the gain calibration of the PMTs. We
have implemented a method of fitting and extracting the position of the single photoelectron peak using the
parameterization described in Ref.~\cite{bellamy1996}.

Figure~\ref{fig:WILLIAM_1} shows the main definitions for the analysis of a representative PMT response obtained
with the LMS to extract the  position of a single photoelectron peak when the average intensity of the LMS light
is about few photoelectrons. Figure~\ref{fig:WILLIAM_2_NEW} shows an example of typical fits of the PMT
response to LMS light of constant intensity at different high voltages. The corresponding dependence of the single
photoelectron peak position as a function of high voltage is given in Fig.~\ref{fig:WILLIAM_3_NEW}. Of course
this method required that the LMS generate light pulses of stable intensity. The results of the single photoelectron
peak position can be used for gain matching. Calibration measurements are performed for all PMTs in parallel at the
same LMS settings. This can be done by adjusting the high voltage applied to individual channels.

\begin{figure}[!h]
    \centering
    \includegraphics[width=1.0\linewidth,trim={0.0cm 0.0cm 0.0cm 0.0cm},clip]{images/WILLIAM_1.png}
    \caption{Definitions: $Q_o$ and $\sigma_o$ are the position and width of the pedestal, $Q_ 1$ is proportional
      to the gain of the PMT. The red curve defines the single photoelectron peak. The green and blue curves describe
      the 2 and 3 photoelectron peaks.}
    \label{fig:WILLIAM_1}
\end{figure}

\begin{figure*}[ht]
\centering
\includegraphics[width=0.99\linewidth]{images/WILLIAM_2_NEW.png}
\caption{Response of a representative PMT to LMS light of constant intensity at high voltage settings from 1560~V
  to 1635~V with a 5~V increment.}
\label{fig:WILLIAM_2_NEW}
\end{figure*}

\begin{figure}[ht]
\centering
\includegraphics[width=0.99\linewidth]{images/WILLIAM_3_NEW.png}
\caption{Single photoelectron peak position as a function of high voltage (V) for a representative HTCC PMT.}
\label{fig:WILLIAM_3_NEW}
\end{figure}

The distinguishing feature of the HTCC LMS is that the observed repeatability of results is within 5-10\% of that
obtained in runs at different but stable light intensities and frequencies. This is demonstrated in
Fig.~\ref{fig:WILLIAM_5} where the positions of the single photoelectron peak at different LMS light
intensities are shown for a representative PMT at a fixed voltage setting. The fitting function is stable across a
wide range of intensities and accurately describes the PMT response at low intensities ($\mu<1.0$). The position of
the peak stays within 5\% of the mean. Consequently there is no need to keep the light source intensity uniform, i.e.
stay the same or close to the same in different calibration runs that are taken whenever necessary.

\begin{figure}[ht]
\centering
\includegraphics[width=0.99\linewidth]{images/WILLIAM_5.png}
\caption{Normalized single photoelectron peak position as a function of LMS light intensity at constant high voltage for a
  representative HTCC PMT. The red lines show a $\pm$5\% deviation, and the blue line shows the averaged
  single photoelectron peak position.}
\label{fig:WILLIAM_5}
\end{figure}

We have compared our preliminary single photoelectron calibration results with those obtained using the
approach described in Ref.~\cite{degtiarenko2017}. The same PMT with a modified divider was tested. Each data
set was normalized by the average value. Figure~\ref{fig:WILLIAM_4} shows the results for the 12 PMTs monitoring
events in the polar angle range from $27.5^\circ$ to $35^\circ$ (labeled as Ring~4) from the 6 sectors. Both approaches
provide consistent results, even though the external light sources and software used in the calibration measurements
were different.

\begin{figure}[ht]
\centering
\includegraphics[width=0.99\linewidth]{images/WILLIAM_4.png}
\caption{Comparison of preliminary calibration results using two different fitting codes from
  Ref.~\cite{bellamy1996} (black dots) and Ref.~\cite{degtiarenko2017} (red dots).}
\label{fig:WILLIAM_4}
\end{figure}

\subsection{Response Equalization}

Different factors (including imperfections of the mirror working surface, dust deposition, condensation of fumes,
overall mirror shape distortions, and gain instability of the individual PMTs) can lead to a variation in the signal
strength from the individual channels, even after comprehensive single photoelectron calibration is complete.
These variations should be corrected independently of their physical origin, as the trigger efficiency is heavily
dependent on the uniformity of the HTCC response. In the beginning of every physics run period we analyze the
first data in order to estimate the signal strength in each of the 48 channels. We then develop corresponding
correction factors, which align the signals between individual channels to the average value between channels. These
correction factors are then propagated to both the offline reconstruction through the CLAS12 calibration database
(ccdb)~\cite{recon-nim} and the online trigger gain files. Figure~\ref{fig:NICK_svodni} shows results for the
channel response before and after equalization.

\begin{figure}[ht]
\centering
\includegraphics[width=0.99\linewidth]{images/NICK_svodni.jpg}
\caption{Response of the PMTs in terms of the number of photoelectrons before and after response equalization.}
\label{fig:NICK_svodni}
\end{figure}

\subsection{Timing Calibrations}

Since the HTCC is the part of the trigger~\cite{trigger-nim}, it is required that the timing of the individual
channels is aligned to aid the online cluster reconstruction. As a result, the timing calibration of the HTCC is done
in two steps: the first step is performed on the level of the FADC, and the second step (finer step) is done in the
offline calibration.

The online calibration is done using the independent trigger from the Forward Tagger detector~\cite{ft-nim}. The
timing of all 48 HTCC channels is aligned in the FADC configuration files by setting the appropriate delays with a
precision of 4~ns (the best available using the FADC). Since the timing resolution of individual channels is on the
order of about 1~ns, we can achieve better resolution of the detector than the 4~ns available from the FADC. The timing is obtained by wave-form analysis of the FADC pulse. We calculate the time at the beam-target interaction vertex for each of the 48 channels and estimate the time
shift between the individual channels. These time shifts are added to the ccdb database and are applied at the
reconstruction stage. Figure~\ref{fig:htcccombinedTimingResponce} shows the timing resolution combined over all
48 PMTs is on the level of 0.6~ns.

\begin{figure}[ht]
\centering
\includegraphics[width=0.95\linewidth]{images/deltaTime6.png}
\caption{Combined timing of the 48 HTCC PMTs. The red curve is a Gaussian fit with a width $\sigma \sim$0.6~ns,
  giving the average system timing resolution.}
\label{fig:htcccombinedTimingResponce}
\end{figure}

\subsection{Event Reconstruction}

The goal of the reconstruction procedure is to reconstruct the real signal strength, time, and hit position from the
raw ADC signals. This is done in two steps:

\begin{enumerate}
\item In the offline decoding stage: the signal is converted from the hardware notation (crate, slot, channel) into the
  CLAS12 notation (sector, layer, component)~\cite{recon-nim}. For each signal, its amplitude (in ADC channels) and
  timing is determined from the threshold crossing, and the pedestal is subtracted. The pedestal value for each
  channel is determined during special pedestal runs and is stored for both trigger and offline reconstruction
  purposes.
\item In the reconstruction stage: the ADC signal is converted into the number of photoelectrons using the
  gain constants in the CLAS12 calibration database (ccdb). The physical design of the HTCC allows the Cherenkov
  radiation from one electron to split into up to four channels (see Fig.~\ref{fig:htccClustersSplit}). In order to
  reconstruct the signal strength, we need to combine such split signals into a single cluster. We start by selecting
  the strongest hit for a given event and use it as the starting point of the cluster. We then look for adjacent hits
  within a certain time window (stored as a parameter in ccdb). If such hits are found, they will be added to
  the cluster. In the final stage, the signal strength is determined as the sum of the individual signals, and the signal
  time is determined as the average between the individual signals, weighted by the corresponding number of 
  photoelectrons. The hit angular coordinate is determined as the average between the individual hits forming the
  cluster. Hits, attributed to the established clusters, are removed from further consideration, and the algorithm
  continues to look for the next cluster until the list of hits is exhausted. See Ref.~\cite{recon-nim} for more
  details.

\begin{figure}[ht]
\centering
\includegraphics[width=0.95\linewidth]{images/htccClustersSplit.png}
\caption{Splitting of the Cherenkov radiation between two mirrors. Geometrically the signal can split into up to four
  mirrors.}
\label{fig:htccClustersSplit}
\end{figure}
\end{enumerate}
