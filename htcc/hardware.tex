\section{Hardware Components and Construction}

\subsection{Lightweight Mirror Prototype Construction}

The multifocal ellipsoidal mirror is the most critical component of the detector. A comprehensive R$\&$D
program was conducted for the purpose of addressing all of the key requirements and to find possible solutions,
and to test and verify the entire technological chain of building mirror prototypes on a 1:2 scale. All core
parameters of the detector were checked and/or derived from the results of Monte Carlo (MC) simulations.

Because there is no mirror support structure, no additional material is introduced  within the acceptance, i.e. in
front of the drift chambers. Therefore the major requirement we needed to satisfy was to build a lightweight
and self-supporting multifocal mirror consisting of many ellipsoidal mirror facets glued together along their
edges. This led to several problems to solve. One of them was to define the contact surfaces of adjacent mirror
facets that would allow final assembly to be completed without any shape adjustments that would leave no gaps
between them. An analytic solution of a system of two second-order equations that describe two intersecting
ellipsoidal surfaces leads to an equation of fourth order in general form. The solutions for any two intersecting
ellipsoidal mirrors of different parameters have been used directly in the design: the line along which two
ellipsoidal surfaces intersect is a flat line of second order, i.e. that which entirely belongs to one particular
well-defined plane. This plane coincides with the edge of each of two adjacent facets that had to be glued
together.

One of the R$\&$D goals was to test this possible solution by building three scaled prototype ellipsoidal facets
that formed a portion of the combined mirror. A complete set of tooling necessary for the thermal shaping of
the front and back films, manufacturing of the ellipsoidal substrates, assembly of the facets, and final trimming
were successfully designed, constructed, and tested. Some parts for tests are shown in
Fig.~\ref{fig:Proto_4parts}.

\begin{figure}[ht]
    \centering
    \includegraphics[width=1.0\linewidth]{images/Proto_4parts.png}
    \caption{Three prototype ellipsoidal foam mirror facets (bottom) and the spherical master table (top).}
    \label{fig:Proto_4parts}
\end{figure}

Three prototype facets (no reflective coatings on the substrates) were put together, touching each other exactly
as designed on the spherical working surface of the assembly table. The back of each facet was of the same
spherical shape. The facets were left under their own weight on the master table of the spherical working surface
for several months, (see Fig.~\ref{fig:Prototype}), to allow checking for any changes in shape and/or quality.

\begin{figure}[ht]
    \centering
    \includegraphics[width=1.0\linewidth]{images/Prototype.png}
    \caption{Three prototype ellipsoidal mirror facets on the spherical master table. They were placed so that they
      would not touch each other. This was in order to leave them free and to check their shape stability.}
    \label{fig:Prototype}
\end{figure}
We learned the following lessons:

\begin{itemize}
\item Each substrate must have a multi-layer structure to avoid post-assembly deformations due to glue
  shrinkage;
\item The thickness of the substrate material had to be between 3/8~in and 3/4~in to satisfy the requirements
  and to stay rigid enough;
\item The substrate material ROHACELL polymethacrylimide foam with a density up to 150~mg/cm$^2$
  could be employed since it has uniform mechanical properties and high radiation resistance;
\item The trimming technology had to be improved to provide increased precision of the mechanical
  processing and final assembly;
\item In all gluing operations non-shrinking glues had to be used or a special technique had to be developed to
  avoid post-polymerization effects;
\item Acrylic films of optical quality have to be used for the front and back surfaces of the mirrors to avoid
  hand-polishing;
\item It is critical that there is structural stability of the mirror facets during the gluing process, which
  includes the complete polymerization time to avoid changes of the geometry of the components.
\end{itemize}

The results obtained were useful in building the final multifocal mirror. Figure~\ref{fig:facet} shows one of the
ellipsoidal facets being prepared for the combined mirror assembly. Final assembly of a 1/12 portion of the full
mirror, consisting of five ellipsoidal coated mirror facets or one ``half-sector'', is shown in
Fig.~\ref{fig:Picture3}.

\begin{figure}[ht]
    \centering
    \includegraphics[width=1.0\linewidth]{images/Picture2.png}
    \caption{An ellipsoidal mirror facet with all four flat contact surfaces prepared for gluing.}
    \label{fig:facet}
\end{figure}
\begin{figure}[ht]
    \centering
    \includegraphics[trim={20cm 15cm 0 10cm },clip,width=\linewidth]{images/Picture3.JPG}
    \caption{Five ellipsoidal mirror facets glued together make up 1/12 of the full HTCC mirror (half of a CLAS12
      forward sector).}
    \label{fig:Picture3}
\end{figure}

\subsection {Ellipsoidal Mirror Facet Manufacturing}

The usage of high-accuracy mechanical processing was absolutely unavoidable in providing the high-precision
mirror facets for the final combined mirror. Putting together 60 ellipsoidal facets of semi-trapezoidal shape,
fitting against each other without adjustment of the overall dimensions of the adjacent facet seams, is a
difficult task. In order to adjust the facets of the combined mirror, each individual facet would need to have
its own support infrastructure, and this would unavoidably introduce additional material. With these concerns in
mind, all of the mechanical processing of the mirror facets was performed with a HAAS 5-axis milling machine.
This allowed us to develop and use special trimming technology for the facets. One such technique was the
``one-shot'' method, which provided the ability from one setting to trim any facet with 3 or 4 contact surfaces
that needed to be glued. This was done to exclude, or at least minimize, the errors introduced when we reset the
orientation of the facets while we cut 3 or 4 edge planes under different combinations of angles. We estimated
that any deviation in the designed dimensions of more than about 0.005~in would make the combined, precision
assembly of so many mirror substrates impossible. This was because any post-manufacturing adjustment of any
of mirror substrates was not an option. In no way could two facets be found to be either overlapping or with
significant gaps between them. These gaps could be as wide as the thickness of a regular glue joint obtained by
simple contact pressure. Otherwise, if these gaps were any larger, they would reduce the working acceptance
and lead to reduced detector efficiency.

The polishing process of large mirrors (8-9~ft diameter) usually means that the manufacturing process is both
labor intensive and time consuming, thus leading it to be very expensive. Therefore, we looked for solutions to
completely avoid any polishing. Due to the fact that we did not require sharp images, the mirror facets were
thus constructed to only work as efficient light collectors. To accomplish the goal we developed and established
an entire assembly procedure, followed by tests of the construction and rating of the final results.

Another issue we addressed was the choice between gluing or mechanical plug-pin assembly procedures. Clearly
gluing introduces deformations due to the shrinkage of any glue. On the other hand, an assembly procedure that
uses location pins results in a more complicated joint since it requires the high-precision processing of plastic foam
parts that are both very lightweight and mechanically weak. Moreover, if any joint deformation was observed after
the first assembly attempt, then many of the parts involved (including the mirror facets) could not be
re-manufactured or used again.

We built 12 identical half-sectors of the combined mirror. Each half-sector consists of 4 ellipsoidal mirrors of
different parameters. The outermost mirror was too large to trim due to the limited travel of the milling machine
table. Therefore, this particular mirror was made of two substrates that were a mirror image of each other.
Consequently each half-sector includes 5 mirror substrates and the full HTCC mirror consists of 60 ellipsoidal
mirror facets in total. 

All mirror facets have the same composite (sandwich) structure: acrylic film (thickness 0.010~in) + foam
(thickness 0.600~in) + acrylic film (thickness 0.010~in). The mirror substrate was made from ROHACELL PMI
(polymethacrylimide) foam and the acrylic films were of optical quality. Manufacturing any substrate was a
multi-stage process:
\begin{itemize}
    \item Thermal shaping of the acrylic film shells for the front (ellipsoidal) and back (spherical) of the mirror;
    \item Manufacturing of the foam substrates;
    \item Assembly (gluing) of the sandwiched mirror substrate;
    \item Trimming of the sandwiched mirror substrate;
    \item Coating of the ellipsoidal faces of the substrate;
    \item Reflectivity tests of the mirror substrate.
    \end{itemize}

Correspondingly we used 5 different high-precision, custom-made tooling sets for the thermal shaping, gluing,
and trimming of the substrates. The thermal shaping was done in a low temperature Precision oven with better
than 0.5$^\circ$C  temperature uniformity in the volume. The tooling set for the manufacturing of mirror facet
\#3, which covers the polar angle in range $\theta = 12.5^\circ - 20^\circ$ and the azimuthal angular interval of
$\Delta \phi = 30^\circ$, is shown in Fig.~\ref{fig:Tool_on_tbl}.

\begin{figure}[ht]
    \centering
    \includegraphics[width=1.0\linewidth]{images/Tool_on_tbl.jpg}
    \caption{Tooling set for the manufacture of mirror facet \#3.}
    \label{fig:Tool_on_tbl}
\end{figure}

The scheme of shaping the acrylic shells is illustrated in  Fig.~\ref{fig:Shaping_new}. The shaping process
was done at temperatures of 105$^\circ$C and differential pressures below 1~atm. There were two
possibilities to shape the shells: use vacuum shaping or just pressurizing the volume up to 2-3~psi differential.
Both options were tried. The option of pressurizing the volume was chosen and used: it provided better control
on the process, i.e. better results. Since the tooling parts were heavy and therefore required a relatively long
time to reach the required temperature, the operations with the oven would take about 3-4~hours, and the whole
process of shaping one shell (load the tooling, heat it up in the oven, cool down to room temperature, and unload
the tooling) would take up to 5~hours.

\begin{figure}[ht]
    \centering
    \includegraphics[width=1.0\linewidth]{images/Shaping_new.png}
    \caption{Scheme of thermal shaping of the spherical shell for the back of the mirrors.}
    \label{fig:Shaping_new}
\end{figure}

The loaded tooling set for shaping the front ellipsoidal shell in the oven by pressurizing the volume above
the film is shown in  Fig.~\ref{fig:Pres_Shaping_Front}. Figure~\ref{fig:Shell} shows the shaped acrylic shell
for the front of the mirror.

%\begin{figure}[ht]
  %  \centering
    %\includegraphics[width=0.95\linewidth]{images/Vac_Mol_Back.png}
    %\caption{Partially loaded tooling set for shaping the spherical back shell of mirror facet \#2.}
    %\label{fig:Vac_Mol_Back}
%\end{figure}

\begin{figure}[ht]
    \centering
    \includegraphics[width=0.95\linewidth]{images/Pres_Shaping_Front.png}
    \caption{Loaded tooling set for shaping the front ellipsoidal shell of mirror facet \#3 in the oven by
      pressurizing the volume with dry air using 1/4-in copper tubing.}
    \label{fig:Pres_Shaping_Front}
\end{figure}

\begin{figure}[ht]
    \centering
    \includegraphics[width=0.90\linewidth]{images/Front_Shell.png}
    \caption{Thermally shaped front ellipsoidal shell for mirror facet \#3.}
    \label{fig:Shell}
\end{figure}

The cut-out of the foam substrate and processing of the back surface of spherical shape was performed without
using any custom-made tools.%, see Fig.~\ref{fig:Cut_Substr}.
%\begin{figure}[ht]
    %\centering
    %\includegraphics[width=0.9\linewidth]{images/Cut_Substr.png}
    %\caption{Milling the back surface of the foam substrate for mirror facet \#3.}
    %\label{fig:Cut_Substr}
%\end{figure}

The front ellipsoidal surface was cut using the tooling that was also used for the final trimming of the sandwiched
glued facet. Once the front surface was cut, the facet was taken off the tooling set for the next operation of
gluing the acryl-foam-acryl sandwich. Figure~\ref{fig:Foam_Sub} shows the fully processed foam substrate for
mirror facet \#3 ready for the assembly of the sandwich. The scheme for the assembly of the sandwiched
substrate is shown in Fig.~\ref{fig:Gluing_Sandwich}. In Fig.~\ref{fig:Assembled_Sandwich} the fully assembled
sandwiched substrate for mirror facet \#3 ready for the final trimming is shown. For better control of the uniformity of glue application (thickness, formation of unwanted bubbles), the first manufactured
mirror facet was assembled using epoxy glue with black dye.

\begin{figure}[ht]
    \centering
    \includegraphics[width=0.9\linewidth]{images/Foam_Sub.png}
    \caption{A completed foam substrate that forms the core of the mirror facet.}
    \label{fig:Foam_Sub}
\end{figure}

\begin{figure}[ht]
    \centering
    \includegraphics[width=0.9\linewidth]{images/Gluing_Sandwich_New.png}
    \caption{Scheme for gluing of the sandwiched mirror substrate.}
    \label{fig:Gluing_Sandwich}
\end{figure}
 
\begin{figure}[ht]
    \centering
    \includegraphics[width=0.9\linewidth]{images/Assembled_Sandwich.jpg}
    \caption{Sandwiched mirror facet after gluing components as shown in Fig.~\ref{fig:Gluing_Sandwich}.}
    \label{fig:Assembled_Sandwich}
\end{figure}

After gluing of the sandwich, it was put back in the tooling set for the final precision trimming. The shells were
designed and processed in a way that allowed unequivocal and simple alignment of all parts during assembly. Each
tooling set had well-defined reference points to provide precise orientation of the set during processing of the
parts. Additionally, the relative position of all tooling components was defined by location pins. Using the same
set of tooling for cutting the face and trimming the facet, guaranteed automatic perfect relative alignment of
the parts being glued together.

Before the trimming operation, the front working face of the substrate was covered with a special tight-fitting
protective film to prevent damage or pollution of the working surface while trimming the substrate. The trimming was done in two steps. In order to
avoid the front shell peeling off the foam, the substrate  was cut through the front shell only using a very small
diameter (0.006~in) end mill, see Fig.~\ref{fig:Trimming_1}. Then the outer portion of the shell was safely
peeled off the substrate and the remaining trimming was performed using a long end mill. The completed final
trimming of the facet is shown in Fig.~\ref{fig:Trimming_2}.

\begin{figure}[ht]
    \centering
    \includegraphics[width=1.0\linewidth]{images/Trimming_1}
    \caption{Cutting through the acrylic shell of the substrate using a small diameter end mill.}
    \label{fig:Trimming_1}
\end{figure}

\begin{figure}[ht]
    \centering
    \includegraphics[width=1.0\linewidth]{images/Trimming_2}
    \caption{Completed final trimming of mirror facet \#3.}
    \label{fig:Trimming_2}
\end{figure}

During final milling the substrate was secured in place by inserting soft foam wedges along the partially cut sides
and glued to the outer portion of the substrate being trimmed. This completely eliminated any vibration that could
ruin the accuracy of the processing. Figure~\ref{fig:Trimmed} shows the completed mirror facet \#3 ready for
deposition of the reflective coating. All substrate trimming was done without any re-positioning of the facet during
the procedure. 

\begin{figure}[ht]
    \centering
    \includegraphics[width=1.0\linewidth]{images/Trimmed}
    \caption{Completely trimmed mirror facet \#3 covered with protection film. }
    \label{fig:Trimmed}
\end{figure}

The face of the mirror substrate did not require any processing before deposition of the reflector material. The
total thickness of the mirror is 130-135~mg/cm$^2$. Acrylic films were glued to both sides of the substrates to
compensate deformation introduced by the shrinking of the thin epoxy glue layers. No shrinkage effects in any of
the produced substrates was observed, thus long-term problems with mirror shape were completely eliminated.
All critical mirror fabrication steps were performed in a clean room (Class 1000). In addition, for better results,
all parts were individually cleaned using an ionizing gun right before assembly. As well, the clean room included a
clean bench with a HEPA air filter next to the assembly table that blew filtered air over the table. Thermal and
mechanical processing was done either with protection films covering critical surfaces or encapsulated in a
gas-tight volume to prevent dust or any other unwanted depositions from damaging the working surface or
otherwise compromising the mirror reflectance.

\subsection{Tests of Mirrors Coated with Reflective Material}

Evaporated Coatings Incorporated (ECI) was chosen from among four potential vendor companies to perform
vacuum deposition of the reflective coating onto the HTCC mirror substrates. Test samples (flat sheets of acryl,
one untouched and one subjected to the same thermal-shaping process used to form the front and back surfaces
of the mirrors) coated by ECI were the most reflective over the entire wavelength range of interest.

For the sample reflectivity tests, a 30~W deuterium lamp was used as an ultraviolet (UV) source from
200-400~nm, and a 50~W quartz-tungsten halogen (QTH) lamp was used as a source of visible light from
370-650~nm. A monochromatic test beam for the reflectivity measurements was generated by a Newport
model 74125 computer-controlled monochromator. A Newport model 10Z40Al.2 flat broadband mirror was
used as a repeatable reference standard for the reflectance measurements. The mirror consists of a
UV-enhanced aluminum coating on a 1/4-in thick, 1-in diameter Zerodur substrate, with a protective overcoat
of UV-transparent magnesium fluoride to prevent oxidation.

The custom coating material used by ECI has an acceptable reflection coefficient in the UV-range and is
resistant to oxidation at room temperatures. Each mirror facet was coated individually along with small flat
acryl sample. The coated acryl sample facet then was tested at the company. The final quality control
measurements of the coated mirror facets was done at Jefferson Lab. Figure~\ref{fig:JLab_Mirror_Better}
shows typical results of the reflectance  measurements of an ellipsoidal mirror facet for the HTCC. The
measured reflectance of the mirror facet (black dots) is very close to the specification shown by the dashed
curve. The reflectance of the reference flat 1-in diameter mirror specified by the vendor and checked at
Jefferson Lab is shown in Fig.~\ref{fig:Ref_Mirror}. The measurement technique has small systematic
uncertainties of about 1-2\%.

\begin{figure}[ht]
    \centering
    \includegraphics[width=1.0\linewidth]{images/JLab_Mirror_Better.png}
    \caption{Typical reflectivity of an ellipsoidal HTCC mirror facet as measured at JLab.}
    \label{fig:JLab_Mirror_Better}
\end{figure}

\begin{figure}[ht]
    \centering
    \includegraphics[width=1.0\linewidth]{images/Ref_Mirror}
    \caption{Typical mirror reference reflectivity  as specified by Newport.}
    \label{fig:Ref_Mirror}
\end{figure}
 
\subsection{Assembly and Tests of Half-Sector Mirrors}

The assembly of the half-sector mirrors was performed on the high-precision half-sector assembly table. The
assembly procedure had to ensure that there were no overlaps or gaps between half-sectors.
Figure~\ref{fig:Half-sector_assem_tb2} shows the table used for the assembly of all 12 half-sector mirrors.
The table was made of solid aluminum alloy block and has several features important for assembly with the
required accuracy:
\begin{itemize}
\item The overall dimensions of the working surface defined the overall dimensions of the half-sector mirror
  being assembled;
    \item The table was equipped with side plates on the left and right of each facet (8 plates total);
    \item The table was designed to be used for gluing of the facets to each other;
    \item The radial and transverse positions of the mirror could be controlled with an accuracy up to 0.001~in by
      inserting spacers between the mirror facets and the side plates;
    \item Each of the 5 places for mounting the different mirror substrates functioned as a vacuum table with a
      spherical work surface, so that each  facet used in the assembly could  be put on the table and secured in place
      as needed by turning on the corresponding diaphragm vacuum pump;
    \item Along the edges of the adjacent mirrors that are in contact, the table has milled-out groves for collecting
      excess epoxy to prevent gluing of the facet to the table surface;
    \item Polymerization of the epoxy glue was possible to perform in a temperature and humidity-controlled
      environment;
    \item The table was part of the setup that allowed for geometry tests of the assembled half-sector.
\end{itemize}

\begin{figure}[ht]
    \centering
    \includegraphics[width=1.0\linewidth]{images/Half-sector_assem_tb2.JPG}
    \caption{High-precision table for the assembly of the half-sector mirrors.}
    \label{fig:Half-sector_assem_tb2}
\end{figure}

The assembly of the combined mirror using location pins is preferable compared with side-to side direct gluing
as there are no deformations involved due to unavoidable epoxy glue shrinkage. Nevertheless, it was decided
not to use location pins since the thickness of the substrates (0.6 in) was relatively small and the mechanical
strength of the PMI foam that we used would introduce risks during final assembly, handling, and installation of
the HTCC mirror. Therefore we decided to directly glue the facets to each other. The gluing technique was
based on applying the glue in the form of dots uniformly distributed over the entire contact surface. The
amount of glue in the dots and the distance between them were such that the glue spots compressed between
the facets did not touch each other. In this case deformations caused by shrinkage of individual dots cancel
each other and the shape of the final product stays unchanged. The only dots that introduce uncompensated
shrinkage deformation are near the edges of the glued surfaces. The corresponding deformations do not change
the mirror shape but they introduce a slight residual waviness of the edges of the mirrors with a pattern that
repeats as the pattern of dots. In fact the only concern was to make sure the glue joints were strong enough.
We ran comprehensive tests to come up with an acceptable solution for using this kind of joint. We tried several
different patterns of glue application, amount and viscosity of the glue, applying glue on one side or both sides,
etc. Figure~\ref{fig:Pattern} shows two identical foam pieces and the epoxy application pattern used in the
tests of the glue joint. 

\begin{figure}[ht]
    \centering
    \includegraphics[width=1.0\linewidth]{images/Pattern.png}
    \caption{Foam pieces marked for an approximately 22.5\% epoxy coverage pattern.}
    \label{fig:Pattern}
\end{figure}

Standard Hysol epoxy with black pigment and 1:1 silica filler to epoxy by volume was applied as small dots
(0.08~in diameter). The viscosity of the epoxy filler mix was so thick that the glue did not bleed into the foam. 

Figure~\ref{fig:Glue_joint_test} shows the test setup to check the strength of the glue joint using two sample
substrate pieces. The epoxy cured for 72~hrs. The total force to break the glue joint was about 62~ft-lbs. The
force was applied evenly and the foam was torn from the glue on both test pieces, see Fig.~\ref{fig:Broken}. The
foam failed and not the glue itself. A set 0.004-in wide gap was left between the parts when gluing, and the glue
was directly applied to one piece only. When the bond failed it pulled out almost all of the glued dots evenly except
for the two places where the shims were set. The foam piece in Fig.~\ref{fig:Broken} that has the dots on it was
the same piece to which the epoxy was applied.

\begin{figure}[ht]
    \centering
    \includegraphics[width=0.95\linewidth]{images/Glue_joint_test.png}
    \caption{Test setup to check the strength of the substrate glue joint.}
    \label{fig:Glue_joint_test}
\end{figure}

\begin{figure}[ht]
    \centering
    \includegraphics[width=1.0\linewidth]{images/Broken.png}
    \caption{Broken epoxy glue joint between two PMI foam pieces.}
    \label{fig:Broken}
\end{figure}

The assembly of half-sectors was done step-by-step by placing mirror facets on the table starting from the
smallest mirror. The first facet once placed on the table was aligned and then checked for fit. After that the
vacuum pump was turned on to secure the mirror. It was impossible to shift the facet on the table by even a
little bit without deforming the mirror once the vacuum was established. The next step was to position the
adjacent mirror with epoxy glue dots on it into position in contact with the first mirror. Once aligned it was
independently secured on the table using the same vacuum pump. Epoxy glue dots were applied on the next
facet and the procedure was repeated until the half-sector was fully assembled.
Figure~\ref{fig:Partial_Half-sector} shows a partially assembled half-sector mirror. The right half (installed)
and left half (yet missing) of the largest mirror have exactly the same geometry.

\begin{figure}[ht]
    \centering
    \includegraphics[width=1.0\linewidth]{images/Partial_Half-sector.png}
    \caption{Partially assembled half-sector mirror. The left facet of the last largest mirror is not installed yet.}
    \label{fig:Partial_Half-sector}
\end{figure} 

Figure~\ref{fig:Half-sector} shows a fully assembled half-sector mirror. It was left on the table under pressure
with vacuum pumps on for at least 24~hours or more depending on the polymerization results of the control glued
samples. 

\begin{figure}[ht]
    \centering
    \includegraphics[width=1.0\linewidth]{images/Half-sector.png}
    \caption{Fully assembled half-sector mirror. The largest mirror consists of two mirror facets that have the
      same geometry.}
    \label{fig:Half-sector}
\end{figure}

The high-precision table was part of the half-sector mirror geometry test setup that was equipped with a
low-energy red laser, gimbal-mounted in the target position, and with four focal planes. The relative locations
of the laser, assembly table, and focal planes were strictly defined by the designed geometry of the HTCC
light collection. The setup allowed for checking the actual geometry of light collection by each mirror using a
point-like laser beam, as well as a beam rastered in the plane crossing any mirror over its entire surface. The
light collection geometry was checked on the half-sector mirrors after complete polymerization of the glue.
The pattern of the light collection obtained on the focal plane of the smallest mirror facet that covers polar and
azimuthal angles of the scattering electrons in the range of $\theta = 5^\circ - 12.5^\circ$ and
$\phi = 0^\circ - 30^\circ$ is shown in Fig.~\ref{fig:Focal_Plane_4}. The concentric circles on the focal plane
are of diameter 1, 2, and 3~inches. Similar results have been obtained for the remaining three mirror facets
covering polar angle ranges of $\theta = 12.5^\circ - 20^\circ$, $\theta = 20^\circ - 27.5^\circ$, and
$\theta = 27.5^\circ - 35^\circ$. The azimuthal angular coverage is the same for all facets.
Figure~\ref{fig:Focal_Plane_1R} shows the light collection pattern for the outermost, largest mirror facet.
All 12 half-sector mirrors were assembled following exactly the same procedures that allowed us to closely
control the overall dimensions and therefore the size of the gaps between adjacent half-sectors. 

\begin{figure}[ht]
    \centering
    \includegraphics[width=0.90\linewidth]{images/Focal_Plane_4.jpg}
    \caption{Light collection pattern on the focal plane. The laser beam is rasterized in the plane crossing the
      mirror covering polar and azimuthal angles in the range of $\theta = 5^\circ - 12.5^\circ$\, and\,
      $\phi = 0^\circ - 30^\circ$, respectively.}
    \label{fig:Focal_Plane_4}
\end{figure}

\begin{figure}[ht]
    \centering
    \includegraphics[width=0.95\linewidth]{images/Focal_Plane_1R.jpg}
    \caption{Light collection pattern on the focal plane for the mirror covering polar and azimuthal angles in
      the range of $\theta = 27.5^\circ - 35^\circ$\, and\, $\phi = 0^\circ - 30^\circ$, respectively.}
    \label{fig:Focal_Plane_1R}
\end{figure}

\subsection{Assembly of the Combined Mirror}

The combined HTCC mirror was assembled based on the experience acquired during the final assembly of the
half-sector mirrors. In order to do this we designed and built the half-sector mirror vacuum holding table for
assembly of the combined mirror. The design of this table and the accuracy of its manufacturing and construction
were critical in providing the required parameters of the combined mirror, such as the geometry of the mirror
optics, the stability of its shape, and its mechanical integrity. A peculiar feature of the combined HTCC mirror is
that it had to provide the correct light collection geometry for all 60 of its mirror facets glued together. The
option of making even very small adjustments of individual facets was excluded by the design.
 
For final mirror assembly we had to build 12 identical half-sector assembly tables and put them together due to
the relatively large overall dimensions of the HTCC mirror. The only difference between the high-accuracy
half-sector assembly table and the 12 identical tables for the final assembly was that they did not have the side
plates. Figure~\ref{fig:One_Foam_Vacuum_Table} shows one of the 12 vacuum tables made of medium density
polyurethane foam with 100\% closed cells.
 
\begin{figure}[ht]
    \centering
    \includegraphics[width=1.0\linewidth]{images/One_Foam_Vacuum_Table.jpg}
    \caption{One of the 12 polyurethane half-sector holding tables.}
    \label{fig:One_Foam_Vacuum_Table}
\end{figure}
 
The top portion of the table is made of one solid block of polyurethane foam. It is glued to a 1-in thick
wedge-shaped flat aluminum plate. To avoid or minimize possible warping we used plates of 1100 aluminum
alloy. The smoothness and accuracy of manufacturing the top surface of the table ensured the ability of the
table to firmly hold the half-sector mirror. No gaskets of any kind were used to enhance the holding ability of
the table. On the working surface of the vacuum table there are five independent circular grooves through
which air is pumped out under each of the five mirror facets of the half-sector mirror. 

The entire set of 12 half-sector vacuum holding tables was assembled on two identical 1-in thick flat plates
carrying 6 tables each. These plates were mounted and aligned on the top of a 10~ft by~10 ft granite table.
Figure~\ref{fig:Twelve_Foam_Vacuum_Tables} shows the vacuum table for the assembly of the combined
HTCC mirror fully equipped with the 60 pumping control valves (5 valves per half-sector). We had to provide
tight ambient control (dust level, temperature, and humidity). The table was also equipped with a transparent
hood (not shown in Fig.~\ref{fig:Twelve_Foam_Vacuum_Tables}) to cover the entire table to run tests at
different relative humidities.   

\begin{figure}[ht]
    \centering
    \includegraphics[width=1.0\linewidth]{images/Twelve_Foam_Vacuum_Tables.jpg}
    \caption{All 12 half-sector vacuum tables assembled on their 1-in aluminum plates placed on the granite table.}
    \label{fig:Twelve_Foam_Vacuum_Tables}
\end{figure}

It was decided not to equip the table with any devices to check the geometry of the HTCC mirror during assembly.
Since the 12 assembled half-sectors passed tight quality controls, there was not much room available for adjustment
of the half-sectors on the final assembly table within more than about 0.030~in in the radial direction and within gaps
between adjacent half-sectors of about 0.010~in. The geometry was essentially established and fixed once all 12
half-sectors were held tight on the table.

The assembly procedure was the same used before for the half-sectors. The only difference was that we had to
install in the center of the combined mirror the lightweight central ring (0.055-in thick) made of carbon fiber.
The ring was glued to all half-sectors. We controlled and measured the gaps between all adjacent mirror facets
belonging to adjacent half-sectors. The average gap was 0.0096~in and is very close to the design value of
0.008~in, which represents the ``dead" zone between the half-sectors. The HTCC covers almost 100\% of the
azimuthal angular acceptance of the CLAS12 Forward Detector. It has to be mentioned that the average gaps
between adjacent facets in a given half-sector mirror were 50\% smaller, which provides for nearly complete
coverage in the polar angle. 

The final assembly of the combined HTCC mirror started with applying the epoxy glue on the first half-sector as
shown in Fig.~\ref{fig:Ap_Gl_Half_Sect}, using the same procedure employed for the assembly of the
half-sectors. Figure~\ref{fig:Partial_Assembl_MIR} shows the partially assembled combined mirror.
% and Fig.~\ref{fig:Compl_Assembl_MIR} shows the completed mirror. 
 
\begin{figure}[ht]
    \centering
    \includegraphics[width=1.0\linewidth]{images/Ap_Gl_Half_Sect.jpg}
    \caption{The dots of epoxy glue being applied to only one side of the half-sector mirror assembly.}
    \label{fig:Ap_Gl_Half_Sect}
\end{figure}
 
 \begin{figure}[ht]
    \centering
    \includegraphics[width=1.0\linewidth]{images/Partial_Assembl_MIR.jpg}
    \caption{ Partially assembled combined HTCC mirror.}
    \label{fig:Partial_Assembl_MIR}
\end{figure}

 %\begin{figure}[ht]
    %\centering
   % \includegraphics[width=1.0\linewidth]{images/%Compl_Assembl_MIR.jpg}
    %\caption{Fully Assembled combined HTCC mirror.}
    %\label{fig:Compl_Assembl_MIR}
%\end{figure}

We used a special procedure to glue the last half-sector because otherwise we would have had to insert the last
half-sector in a very narrow space that would have smeared the glue dots. Therefore, we assembled the first 6
half-sectors on the first 1-in mounting plate, and the remaining 6 half-sectors on the other 1-in mounting plate.
The mounting plates with the 6 half-sectors were positioned on the granite table leaving a gap about 1-in wide
(see Fig.~\ref{fig:Separated_halves}). Epoxy glue was then applied to the one of the exposed sides, (see
Fig.~\ref{fig:Final_Gluing}), and the plates were slid together so that both sides come in contact.
Figure~\ref{fig:Compl_Assembl_MIR} shows the completed mirror. 
 
  \begin{figure}[ht]
    \centering
    \includegraphics[width=1.0\linewidth]{images/Separated_halves.jpg}
    \caption{Separated halves of the combined mirror before final gluing.}
    \label{fig:Separated_halves}
\end{figure}
         
 \begin{figure}[ht]
    \centering
    \includegraphics[width=1.0\linewidth]{images/Final_Gluing.jpg}
    \caption{Application of the epoxy glue on the side of the separated halves of the combined mirror.}
    \label{fig:Final_Gluing}
\end{figure}
  
 \begin{figure}[ht]
    \centering
    \includegraphics[width=1.0\linewidth]{images/Compl_Assembl_MIR.jpg}
    \caption{Fully assembled combined HTCC mirror.}
    \label{fig:Compl_Assembl_MIR}
\end{figure}

All elements that support and hold the combined mirror are out of the acceptance of the HTCC (see
Fig.~\ref{fig:Support_Ring}). The rigid, lightweight, composite supporting ring (strong-back) was attached to
the combined mirror via 12 composite lightweight bridge pieces glued to the sides around of the mirror. A
rendering of the completely assembled HTCC mirror, ready for installation, is shown in
Fig.~\ref{fig:Ring_to_Mirror}. Since the rigidity of the supporting parts is much higher than the rigidity of
the combined mirror, we used flexible silicon compound for gluing.
  
\begin{figure}[ht]
    \centering
    \includegraphics[width=1.0\linewidth,trim={0 5cm 0 0},clip]{images/Support_Ring.jpg}
    \caption{Supporting elements ready to be attached to the combined mirror. The mirror is covered with soft
      paper towels to protect the working surface from debris.}
    \label{fig:Support_Ring}
\end{figure}

\begin{figure}[ht]
    \centering
    %\includegraphics[width=1.0\linewidth,trim={0 5cm 0 0},clip]{images/Ring_to_Mirror.jpg}
    %\includegraphics[width=1.0\linewidth,trim={0 5cm 0 0},clip]{images/Support_Ring_2.jpg}
        \includegraphics[width=1.0\linewidth]{images/Support_Ring_2.jpg}
    \caption{Completely assembled HTCC mirror ready for installation.}
    \label{fig:Ring_to_Mirror}
\end{figure}
 
\subsection{HTCC Gas Volume Entry and Exit Windows}

There are several aspects that have been taken into consideration that define the design of the entry and
exit windows:

\begin{itemize}
    \item Large area to cover;
    \item Small thickness;
    \item Opaque;
    \item High durability;
    \item Attachment to the main frame;
    \item Structural stability, i.e. resistance to pressure variations.
    \end{itemize}

The entry and exit windows are composite films made of three layers laminated together: Tedlar (thickness
38~$\mu$m), Mylar (thickness 75~$\mu$m), Tedlar (38~$\mu$m). The composite films came in rolls 61-in
wide. To make the exit window, three composite films were glued together side by side. The glue joint between
adjacent composite films was made in such a way that the thickness of the joint exceeded the remaining portions
by no more than 10\%. The usage of two black Tedlar films in the composite window guaranteed light insulation
even if one layer had any holes. One layer of Mylar film provided excellent durability and flexibility.

The dimensions of the entry and exit windows are $\approx$2.5~ft and $\approx$9.5~ft, respectively, so the
difference is significant. This required developing a special design for their attachment to the body of the
detector. The primary electron beam passes through the HTCC exactly along the axis of the detector. To
decrease the background of M{\o}ller electrons we have used a long shielding piece made of tungsten around
the beam that nominally covers polar angles up to 2$^\circ$ and has a small cylindrical opening in the center that
goes all the way through and is big enough for the beam~\cite{beamline-nim}. The volume of the HTCC must be
separated from the volume occupied by the tungsten metal shield. Since the corresponding HTCC part called the
M{\o}ller Cup that is concentric with the tungsten absorber must be lightweight, the joints between this part
and the entry and exit windows must also be lightweight. In this case, since the windows have different
dimensions, any changes in atmospheric pressure would cause both windows and the M{\o}ller Cup attached to
them to move upstream or downstream - depending on atmospheric pressure changes. The mirror could be
damaged by the exit window if the pressure goes up, or it could be damaged by the conical M{\o}ller Cup if the
pressure goes down (see Fig.~\ref{fig:side_view}). Thus the M{\o}ller Cup has to be kept in the same location
relative to the mirror regardless of the fluctuations in atmospheric pressure. Even small changes of
$\sim$1~mm of Hg would generate a force of $\sim$200~lbs acting on the M{\o}ller Cup along its axis.  

\begin{figure}[ht]
    \centering
    \includegraphics[trim={1.5cm 5cm 0 2cm }, clip, width=\linewidth]{images/Spokes2.png}
    \caption{Side view of the HTCC. The  entry and exit windows are shown along with other internal components.
    The beam is incident from the left.}
    \label{fig:side_view}
\end{figure}

To avoid potential problems with the integrity of the detector, the M{\o}ller Cup was attached to the main frame
of the HTCC at 12 points: 6 points on the upstream portion of the main frame, and the remaining 6 points on the
downstream portion. All parts providing attachments of the M{\o}ller Cup to the front or to the back of the main
frame are completely located in the shadow zone of the 6 superconducting coils of the torus magnet
\cite{magnets-nim}, i.e. they do not create any obstruction to the particles going through the drift chambers.
The M{\o}ller Cup was attached using 12 thin spokes, each 1.5~mm in diameter and was made of carbon fibers to
minimize the possible scattering of particles traveling within the shadow of the torus coils. The spokes very
firmly hold the M{\o}ller Cup in position. Each of them was tensioned as necessary to provide structural rigidity
and to withstand the stresses generated by the attached windows during atmospheric pressure changes. They
were tensioned as a string in order to eliminate any possible damage due to deformations of the body of the
HTCC while transporting, installing, or aligning. Each spoke is spring-loaded from both ends.
Figures~\ref{fig:Front_View} and ~\ref{fig:Exit_Win} show the upstream and downstream views of the HTCC
with the entry and exit windows installed.

\begin{figure}[ht]
    \centering
    \includegraphics[width=1.0\linewidth,trim={0 2.5cm 0 0},clip]{images/Front_View}
    \caption{Upstream view of the HTCC.}
    \label{fig:Front_View}
\end{figure}

\begin{figure}[ht]
    \centering
    \includegraphics[width=1.0\linewidth,trim={0 25cm 0 500},clip]{images/Exit_Win.jpg}
    \caption{Downstream view of the HTCC.}
    \label{fig:Exit_Win}
\end{figure}

\subsection{Containment Vessel and Combined Mirror Installation}

The HTCC Containment Vessel has properties to satisfy a number of requirements, which included the safe
transportation of the fully assembled HTCC to the experimental hall without any changes in the alignment of the
internal components and the preservation of the mirror integrity. We tested the integrity of the spare mirror
(see Fig.~\ref{fig:transportation_spare_mirror}) by transporting it along the chosen route. We successfully
transported the detector using the results obtained during the test. 

\begin{figure}[ht]
    \centering
    \includegraphics[trim={1.5cm 5cm 0 2cm }, clip, width=\linewidth]{images/Road_Test.JPG}
    \caption{Road tests of the spare mirror.}
    \label{fig:transportation_spare_mirror}
\end{figure}

The vessel had to be rigid, have negligible deformation while changing its orientation and, at the same time, allow
easy access to any internal components. There is one special requirement for the mirror support structure. The
Containment Vessel has only a limited rigidity and even small deformations could directly lead to a dangerous
deformation of the HTCC mirror if it was attached to the Containment Vessel directly in ordinary ways. Even if
the mirror remains whole and without any cracks, the light collection pattern could still be changed and decrease
the signal strength.

In general the vessel works as the support structure for all internal components and must be both light-tight and
gas-tight. Safety considerations require that we have both easy and safe access to the components inside. This is
absolutely necessary during maintenance and while running alignment checks. Special attention was paid to the
routing of cable and fiber optics inside the volume. These items are very difficult to replace. The vessel is
equipped with a local gas distribution and a control panel. The control panel is for safe and continuous purging
of the volume with different dry gases (as needed) and used to keep the water vapor concentration level under
tight control during both operations and maintenance.

There was a need to have easy access to any of the 48 photomultiplier tubes (PMTs) to adjust their alignment
and for maintenance. The Containment Vessel has 24 service hatches wide enough to perform work on any channel.
Each channel consists of a PMT with high voltage divider, magnetic shield with compensation coil, and Winston
cone, which are installed in the PMT mounting fixture together as one unit, (see Fig.~\ref{fig:PMT_Mount}).

\begin{figure}[ht]
    \centering
    \includegraphics[width=1.0\linewidth,trim={0 12cm 0 0},clip]{images/PMT_Mount.jpg}
    \caption{PMT mounting unit with components.}
    \label{fig:PMT_Mount}
\end{figure}

Checks of the cabling and installed fiber optics used for calibration of the PMTs can also be done using the
service hatches. Service work can be performed on the detector while it is in its nominal working location. Any
access to the internal components of the detector requires replacement of the working gas with dry air. For
safety a procedure of purging the HTCC volume has been established to allow access only when the concentration
of oxygen in the volume exceeds 19.5\%.

The combined HTCC mirror is supported and held in the Containment Vessel by 6 orthogonal links. These links
connect the supporting ring (attached to the mirror) to the Containment Vessel. It was critical that any
deformation of the Containment Vessel not be transmitted to the mirror. Each link has a ball-end swivel on each
end. By using the minimum number of links (6) to constrain all motion, the mirror could be aligned, but no forces
above those due to gravity on the mass of the mirror and its strong-back are ever placed on the mirror. The set
of links are attached to the Containment Vessel at 3 points that are spaced 120$^\circ$ around the perimeter
of the ring. This scheme of attachment was tested using a very lightweight 5-ft diameter flat mirror. The tests
showed that the light collection pattern stays unchanged within a sufficiently wide range of deformations of the
frame that supports the mirror. Therefore possible deformation of the Containment Vessel during installation
and alignment do not affect the original shape of the HTCC mirror. Figure~\ref{fig:HTCC_MIRR_INST_NEW}
shows the HTCC mirror installed in the Containment Vessel.

\begin{figure}[ht]
    \centering
    \includegraphics[width=1.0\linewidth,trim={0 0cm 0 0},clip]{images/HTCC_MIRR_INST_NEW.jpg}
    \caption{Combined mirror installed in the Containment Vessel. The set of links holding the mirror in position
      allow for fine adjustment with an accuracy of $\sim$0.01~in or better.}
    \label{fig:HTCC_MIRR_INST_NEW}
\end{figure}

The HTCC is susceptible to noticeable deformations due to the large overall dimensions of the detector. The
light and gas leak protection measures provided thus had to be reliable and require little maintenance. All of
the inside surfaces of the Containment Vessel have been painted a flat black to reduce light reflectance, and
all of the borders between adjacent parts that form the outside shell of the detector have been sealed with
a flexible black silicone gel on both the inside and outside of the vessel. Sealing all of the inside seams was
necessary to allow the detector to always stay under a positive differential pressure during variations in the
atmospheric pressure. As a result of even small changes in the differential pressure, the vessel would be
deformed due to its large volume, i.e. the pressure is applied to a large surface area. 

\subsection{Alignment of the Light Collection Components}

The HTCC contains light collection and light detection components: the mirror, Winston cone light concentrators,
and photomultiplier tubes (PMTs). Even if the mirror is constructed and installed properly as designed, final
checks of the component alignment are needed. We have conducted comprehensive checks of the light collection
optics on the fully assembled detector. This work was done before the detector was moved to the experimental
hall. For the alignment checks we again used a low-power laser, gimbal mounted in the target position. To operate
the laser we used a set of standard high-precision devices to control the position and orientation of the laser.
The opaque entry window was replaced with a thin transparent film in order to keep the volume of the detector
isolated as much as possible. We opened one access hatch at the time for short periods of time to install
templates on the face of the accessible Winston cones and perform adjustments of the housing units each
containing a 5-in PMT, Winston cone, 3-layer magnetic shield, and compensation coil. The alignment of all 48
PMT housing units was checked and adjusted as needed.

Each mirror facet was illuminated with the laser at 5 points: the center of the facet and its four corners, and the
reflected light  pattern was photographed. For some of the channels we checked the light collection geometry at
normal relative humidity in the HTCC volume and at 0\% relative humidity. 

Figure~\ref{fig:GEO_TEST_3_Normal} shows the pattern of the light reflection when mirror facet \#3 was
illuminated in the center. Circles of diameter 1~in, 3~in, and 5~in concentric to the PMT are shown. The result was
obtained at normal relative humidity. Results obtained at RH=0\% for the same channel show small but acceptable
differences (see Fig.~\ref{fig:GEO_TEST_3_Zero}). Similar geometry test results were obtained for channel
\#4 covering polar angles in range of $5^\circ$ to $12.5^\circ$. They are shown in 
Figs.~\ref{fig:GEO_TEST_4_Normal} and ~\ref{fig:GEO_TEST_4_Zero} obtained at different relative
humidities.

Considering the light collection patterns obtained when the mirror facets were illuminated in the corners we made
the necessary adjustments in the alignment of the PMT mounting units. No adjustments were needed for the HTCC
mirror. Figure~\ref{fig:Ch_5_1_3_Before_NEW} shows the test results for mirror facet \#3 from sector 5,
half-sector 1 obtained before adjustments in alignment were done. Figure~\ref{fig:Ch_5_1_3_After_NEW} shows
the changes in the light collection pattern after the alignment adjustments. The image has been shifted toward the
center. Figure~\ref{fig:GEO_TEST_5_1_3_Center} shows a photograph taken when the mirror was illuminated in
the center. The five circles concentric to the PMT shown in the photograph have diameters from 1 to 6~in. 

\begin{figure}[ht]
    \centering
    \includegraphics[width=1.0\linewidth,trim={0 8.5cm 0 0},clip]{images/GEO_TEST_3_Normal.jpg}
    \caption{Geometry test result for channel \#3 covering polar angles from $12.5^\circ$ to $20^\circ$ at
      nominal relative humidity. The corresponding mirror facet was illuminated in the center.}
    \label{fig:GEO_TEST_3_Normal}
\end{figure}

\begin{figure}[ht]
    \centering
    \includegraphics[width=1.0\linewidth,trim={0 8.5cm 0 0},clip]{images/GEO_TEST_3_Zero.jpg}
    \caption{Geometry test result for channel \#3 covering polar angles from $12.5^\circ$ to $20^\circ$
      at RH=0\%. The corresponding mirror facet was illuminated in the center.}
    \label{fig:GEO_TEST_3_Zero}
\end{figure}
        
\begin{figure}[ht]
    \centering
    \includegraphics[width=1.0\linewidth,trim={0 8.5cm 0 0},clip]{images/GEO_TEST_4_Normal.jpg}
    \caption{Geometry test result for channel \#4 covering polar angles from $5^\circ$ to $12.5^\circ$ at
      nominal relative humidity. The corresponding mirror facet was illuminated in the center.}
    \label{fig:GEO_TEST_4_Normal}
\end{figure}

\begin{figure}[ht]
    \centering
    \includegraphics[width=1.0\linewidth,trim={0 8.5cm 0 0},clip]{images/GEO_TEST_4_Zero.jpg}
    \caption{Geometry test result for channel \#4 covering polar angles from $5^\circ$ to $12.5^\circ$ at
      RH=0\%. The corresponding mirror facet was illuminated in the center.}
    \label{fig:GEO_TEST_4_Zero}
\end{figure}

\begin{figure}[ht]
    \centering
    \includegraphics[width=1.0\linewidth,trim={0 0cm 0 0},clip]{images/Ch_5_1_3_Before_NEW.jpg}
    \caption{Geometry test result for sector 5, half-sector 1, mirror \#3 covering polar angles in range of
      $12.5^\circ$ to $20^\circ$ obtained before adjustment when mirror \#3 was illuminated in the center (blue),
      and its corners (purple). The black circles are 1~in to 6~in in diameter. The circle (red) is of radius 0.45~in,
      and is equal to the RMS of the fitted center of gravity of the light collection pattern.}
    \label{fig:Ch_5_1_3_Before_NEW}
\end{figure}

\begin{figure}[ht]
    \centering
    \includegraphics[width=1.0\linewidth,trim={0 0cm 0 0},clip]{images/Ch_5_1_3_After_NEW.jpg}
    \caption{Geometry test result for sector 5, half-sector 1, mirror \#3 covering polar angles in range of
      $12.5^\circ$ to $20^\circ$ obtained after adjustment when mirror \#3 was illuminated in the center (blue) and
      its corners (purple). The black circles are 1~in to 6~in in diameter. The circle (red) is of radius 0.51~in, and is
      equal to the RMS of fitted center of gravity  of the light collection pattern.}
    \label{fig:Ch_5_1_3_After_NEW}
\end{figure}

\begin{figure}[ht]
    \centering
    \includegraphics[width=1.0\linewidth,trim={0 0cm 0 0},clip]{images/GEO_TEST_5_1_3_Center.jpg}
    \caption{Geometry test result for the sector 5, half-sector 1, mirror \#3 covering polar angles in range of
      $12.5^\circ$ to $20^\circ$ obtained after adjustment when mirror \#3 was illuminated in the center.}
    \label{fig:GEO_TEST_5_1_3_Center}
\end{figure}

Figure~\ref{fig:subfig} shows the final light collection patterns obtained for all mirrors for half-sectors 1 and
2 from sector~5. One can clearly see that the light collection is more focused for the small mirrors than for the
large ones. This effect is caused by the difference in the rigidity between the combined mirror itself and the
supporting ring attached to it, as well as the different sensitivities to the changes in relative humidity of the
environment.
 
Since the mirror has a funnel shape, see Fig.~\ref{fig:Colored_Mirror}, the nose of the funnel gets stretched
less than outer portion of the mirror. The outer portion of the mirror consists of mirror facets \#1 and \#2, the
largest mirror facets. Even though the widths of facets \#1 and \#2 measured in the radial direction are close to
each other, the effect of humidity changes for the outermost facet \#1 is larger than for facet \#2 due to the
funnel shape of the combined mirror.

\begin{figure*}
\begin{subfigure}[b]{0.49\textwidth}
    \includegraphics[width=1.0\textwidth]{images/GEO_TEST_Sect5_M_1_M_2.jpg}
    \caption{} \label{fig:subfig1_a}
\end{subfigure}
\hspace*{\fill} % separation between the subfigures
\begin{subfigure}[b]{0.5\textwidth}
    \includegraphics[width=1.0\linewidth]{images/GEO_TEST_Sect5_M_3_M_4.jpg}
    \caption{} \label{fig:subfig1_b}
\end{subfigure}
\caption{Light collection test result for both half-sectors of sector 5. (a) - Mirrors \#1 and \#2 cover polar
  angles in the range of $20^\circ$ to $35^\circ$  within an azimuthal interval of $60^\circ$. (b) - Mirrors \#3 and
  \#4 cover polar angles in range of $5^\circ$ to $20^\circ$  within an azimuthal interval of $60^\circ$. The data
  are shown after adjustment of the light-collection optics. The mirrors were illuminated in the center and the
  corners.} 
\label{fig:subfig}
\end{figure*}

\begin{figure}[ht]
    \centering
    \includegraphics[width=1.0\linewidth,trim={0 0cm 0 0},clip]{images/Colored_Mirror.jpg}
    \caption{The combined HTCC mirror is funnel-shaped.}
    \label{fig:Colored_Mirror}
\end{figure}

\subsection{Gas Composition Control}

The fully assembled detector was tested for gas and light leaks. Gas tightness was checked by filling the volume
with a mixture of dry air and a small amount of non-flammable gas at positive differential pressure. Freon gas leak
sniffers were used. As expected, most of the leaks were found around the entry and exit windows because both
windows were sealed only from the outside as they were the last two components attached to the Containment
Vessel. Light tightness was checked by monitoring the counting rates of the photomultiplier tubes while illuminating
the sealed seams of the vessel with an external light source. The rates were close to the dark counting rates whether
the lights in the hall were turned on or off. The gas tightness was controlled by fixing the leaks found and then by
measuring the humidity inside the detector while it was continuously purged with dry nitrogen. At flow rates of
10 - 15~l/min, a humidity level not higher than $\sim$100~ppm was measured in a span of 2-3~days of continuous
purging. These results can be used as a good indication of a low level of water vapor and oxygen content. Note that
the diffusion of water vapor and oxygen through the windows is defined by the Tedlar films, which have the lowest
diffusion coefficients. For Tedlar the diffusion of oxygen is lower than the diffusion of water vapor. The presence of
water vapor and oxygen in the working volume is unavoidable but should be kept at the lowest possible level because
water vapor and CO${_2}$ gas produce carbonic acid that may be harmful to the mirror working surface. As far as
oxygen content is concerned, it also needs to be kept at the lowest possible level. During operations with beam a
certain amount of ozone can be generated due to radiation. Both oxygen and ozone absorb the ultraviolet component
of Cherenkov light in the HTCC generated by scattered electrons. Consequently the signal strength becomes lower,
which directly leads to a reduced electron detection efficiency and a reduced rejection of charged pions.

Another reason to keep tight control on humidity is related to the sensitivity (to some extent) of the mirror shape
to humidity. The mirror must be used at the lowest humidity level, but the entire manufacturing process of the
mirror facets was done at normal room conditions. Once the mirror had been installed, the HTCC volume was sealed
and purged with dry gas (N${_2}$, CO${_2}$, or dry air). Altering the humidity from almost zero to normal
atmosphere conditions may lead to component fatigue and cause structural deformation. During maintenance the
volume is partially exposed to the outside environment, which increases humidity. In any case, all maintenance
activities are stopped once the relative humidity inside the volume reaches 2-3\%. This is controlled at the exhaust
of the detector. Maintenance is resumed only after the operational humidity level is restored.

The detector is equipped with a local gas control panel. The parameters that can be read directly are limited to the
pressure at the input of the volume and the differential pressure. The HTCC is continuously monitored online by a
system that monitors the following parameters:

\begin{itemize}
    \item Type of gas;
    \item Gas flow rate;
    \item Differential pressure;
    \item Humidity;
    \item Amount of gas already consumed.
\end{itemize}

The online control system allows the detector to be safely operated within predefined intervals of parameter
variations. This system generates warnings and alarms that require either remote or in situ response. In the case of
a power outage in the hall, the mass flow controller turns off and purging is stopped. It takes several hours for the
humidity to rise up to 2-3\% due to direct leaks and diffusion of the ambient humid air inside the working volume.
This provides enough time for operation to switch the gas to a manual bypass rota-meter, which is normally closed.
The local gas panel includes three specialized filters that prevent dust, water vapor, and oil vapor from entering
the volume.
